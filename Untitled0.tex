%Options: KBD=VOID; STD=VOID; LANG=ENGLISH; FMT=LATEX; HYPHEN=default;
\documentclass[a4paper,12pt]{article}
\usepackage[T1]{fontenc}
\usepackage[ansinew]{inputenc}
\usepackage[frenchb]{babel}
\usepackage{color, graphics, graphicx}
\usepackage[vtex]{hyperref}
\usepackage{MarvoSym}
\textwidth 17cm
\textheight 23 cm
\topmargin -1cm
\oddsidemargin -1cm
\parindent 0pt
\begin{document}

\textbf{Proposition de paragraphe de remplacement.}(J'ai bien entendu travaill� en radians!)  \bigskip

Ayant calcul� ci-dessus  les valeurs de ${\mathcal S}_{\hbox{cube}}$ et  ${\mathcal
S}_{\hbox{t�tra�dre}}$, la question � �tudier devient:
\textit{la relation      $m\pi + n\alpha =       k\pi$ est-elle satisfaite pour des valeurs
enti�res naturelles de $m$, $n$ et $k$?} \bigskip


Notons que  les param�tres $m$ et $k$  d�pendent des d�compositions du cube et du t�tra�dre.  Quant au
param�tre  $n$, il d�pend �galement de la d�composition du t�tra�dre, mais nous savons qu'il n'est pas nul et
  nous avons fait en sorte que $n\alpha$ soit inf�rieur � $\pi$. Et comme $\alpha = \arccos \frac{1}{3} \simeq
1,23\ldots$ le naturel $n$ ne peut d�passer $\frac{\pi}{\alpha}=2,55\ldots$ Il suffira donc d'�tudier les
cas $n=1$ et $n=2$.
 

Le rapport $\frac{\pi}{\alpha}$ m�rite quelque attention. En effet la relation ci-dessus peut s'�crire
$\frac{n}{k-m}=\frac{\pi}{\alpha}  $. Si elle �tait vraie, le rapport $\frac{\pi}{\alpha}$ serait rationnel.
Cela vaut la peine de le signaler!

Reprenons la relation �tudi�e sous la forme $n\alpha = (k-m)\pi$.

 $n$ ne peut valoir 1 car la relation
$\arccos \frac{1}{3} =  (k-m)\pi $ entra�ne $\frac{1}{3} = \cos (k-m)\pi = \pm1$

Supposons $n = 2$. Alors     $ \cos (2\alpha) = 2\cos^2(\alpha)-1 = -\frac{7}{9}$ et � nouveau, la relation
est fausse.


\end{document}
