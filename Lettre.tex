%Options: KBD=VOID; STD=VOID; LANG=ENGLISH; FMT=LATEXL1; HYPHEN=default;
\documentclass[a4paper,12pt]{article}
\usepackage[T1]{fontenc}
\usepackage[ansinew]{inputenc}
\usepackage[frenchb]{babel}
\usepackage{graphicx}
\textwidth 16cm
\textheight 23cm
\oddsidemargin 0cm

\begin{document}

Cher(e) Coll�gue,\vspace{2\baselineskip}


Lors de l'assembl�e g�n�rale du CREM, vous avez accept� de tester la version 2.4.0 d'Apprenti G�om�tre, en particulier des   fonctionnalit�s relatives aux macros.
Je vous en remercie vivement.

Le logiciel est disponible en versions Windows, Linux et MacOsX. Vous pourrez t�l�charger la version de votre choix conform�ment aux indications donn�es en annexe.
Veillez � t�l�charger simultan�ment le guide utilisateur qui vous donnera les indications n�cessaires. Vous pouvez le consulter soit directement (c'est un fichier pdf) soit en cliquant sur le menu
\og Aide/Guide Utilisateur\fg\ du logiciel. Installez l'int�gralit� du package dans un seul r�pertoire, r�serv� � cette version.

Lorsque  vous constatez un probl�me, ou lorsque vous souhaitez formuler des suggestions, le plus simple pour vous comme pour nous est que vous remplissiez
le formulaire accessible via le menu Beta (le dernier � droite) du logiciel.   Un clic sur le bouton \og Envoyer\fg, en bas � droite, et c'est termin� (si vous �tes connect�).

En vous souhaitant d'excellentes vacances, avec ou sans Apprenti G�om�tre, je vous donne rendez-vous, probablement lors du congr�s de la SBPMef.

Bien cordialement
\vspace{2\baselineskip}

G. No�l
\end{document}
